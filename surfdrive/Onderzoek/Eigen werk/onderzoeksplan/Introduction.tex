\section{Introduction}\label{sec:intro}
Assessment is one of the most fundamental aspects of education, playing an integral role in teaching and learning. In particular, summative assessment is a key indicator for evaluating the extent to which learning objectives are being met. Adequately determining a student's proficiency of concepts, skills and competencies is a complex undertaking. Valid, objective, reliable, and fair assessments call for a setting in which students are given ample opportunity to provide evidence of their knowledge and skills in such a manner that a teacher can effectively observe and measure their level of mastery\todo{attainment level reached}. Computer science is a multifaceted subject, bringing together conceptual knowledge, computational thinking, and problem-solving skills. To assess achievement levels on its crossroads is far from trivial. In addition to pedagogical and content knowledge, it requires knowledge on constructing frameworks, test specifications, and the ability to align according to specific classroom needs. This research explores the characteristic features of effective CS assessment, which incorporate the wide range of knowledge, computational thinking and problem solving skills, and what pedagogical-content-knowledge teachers need for its implementation. 