\section{Introduction}\label{sec:intro}
Assessment is one of the most fundamental aspects of education, playing an integral role in teaching and learning. In particular, summative assessment can be used for gauging to which extent learning objectives are being met. Adequately determining a student's proficiency of concepts, skills and competencies is a complex undertaking. Valid, objective, reliable, and fair assessments call for a setting in which students are given ample opportunity to provide evidence of their knowledge and skills, and that a teacher can effectively observe and measure their level of mastery\todo{attainment level reached}.

Computer science is a multifaceted subject, bringing together conceptual (CS) knowledge, computational thinking, and problem-solving skills. This knowledge can be categorized as factual knowledge (knowing what), procedural knowledge (knowing how), conceptual knowledge (knowing why), and meta-cognitive knowledge (knowing about knowing)\cite{streun2001kennis}. To assess achievement levels on its crossroads is far from trivial. In addition to pedagogical and content knowledge, it requires knowledge on constructing frameworks, test specifications, and the ability to align learning objectives with assessment tasks which match specific classroom needs. This research explores the characteristic features of effective CS assessment, which incorporate the wide range of knowledge, computational thinking and problem solving skills, and what pedagogical-content-knowledge teachers need for its implementation.

\subsection{Current Situation}


As of 2019, a reformed Computing Science curriculum will become effective in the Netherlands. The reformed program follows in the footsteps of recent international reforms (examples are the U.S. CSTA, U.K. National curriculum, New Zealand's new "Digital Technologies") where a shift from computer and ICT applications towards rigorous computing has taken place. \citeA{Brown2013} report on challenges and successes of implementation in the UK. \citeA{Bell2014} discuss issues that teachers face after New Zealand's new standards were introduced in 2011. On another continent, \citeA{Yadav2016} discuss the implications and challenges on CS teachers in the Unites States.

In the Netherlands, the reform has resulted in a stronger focus on fundamental concepts, including analysis of standard algorithms and being able to compare efficiency and implementability \cite{Barendsen2016}. Higher-order (Computational Thinking) skills are positioned as a fundamental pillar, characterized as one of the 21st-century skills competencies (\cite{SLO2015}). The idea of Computational Thinking (CT) presents a different way of thinking about computing proficiency by emphasizing the (problem-solving) skills that students need to solve complex real-world problems and evaluate proposed solutions. It also implies that solely measuring CS proficiency as the acquisition of core content knowledge no longer suffices \cite{Yadav2015}. Despite the proliferation of CT-focused programs throughout several curricula around the world, there is still no general method\todo{Erik: is die er wel in andere vakken?} on how to develop CT assessments reliably and easily \cite{catete2017framework}. In their extensive review study, \citeA{LyeKoh2014} find a few studies that have examined computational skills as outcomes. No studies were found focussing on upper secondary high school. We were able to locate one: \citeA{Lee2011}. All-in-all, little research has been done on the effectiveness of the assessment materials available \cite{Yadav2016}.
%How will this fundamental shift evolve (in) the Dutch educational landscape?\todo{Erik: kopje? wat bedoel je hiermee Erik?}




Currently, only a small percentage of the Dutch CS teachers has completed a full academic qualification to teach the subject \cite{tolboom2014informatica}.

New curricular material comes with new misconceptions. \citeauthor{duncan2017teachers}'s results confirm their concern and warn that insufficiently prepared teachers may not pick up student misconceptions, or may even introduce incorrect ideas themselves. Speculating on their content knowledge (as a component of the PCK), Dutch teachers currently focus primarily on designing and implementing computer systems and programs \cite{Schmidt2007}, rather the more fundamental underlying concepts which the new curriculum now focusses on. In the Netherlands there is no centralized or standardized exam for the subject, no validated assessment tasks, and no way to ascertain the effectiveness of teaching. Teachers must develop their own assessments. From their community of practice observations, \citeA{hermansen2014reworking} conclude that teachers create their own assessments, generally by adapting others. There is usually only one teacher in a school, leaving them isolated from their peers. Compared to initiatives such as the U.K. Computing At School (CAS), professional development opportunities, communities of practice, sharing of experiences, resources, and lessons learned are sparse. All in all, teachers have no assessment benchmark, and lack valid and reliable example assessments \cite{Yadav2015}, and furthermore lack evaluation or reflection on their teaching strategies via peer-collaboration. In their research, \citeA{GroverPea2013} describe that paying attention to assessment is a pre-requisite to successful implementation of any curriculum and conclude that large gaps exist pertaining to assessment. Denning \cite{denning2017remaining} notes that educators internationally have many basic questions about teaching and assessing CT and expresses his concerns for ineffective teaching. Along the same line, \citeA{2013Seiter} describes assessment models incorporating progression pathways as vital for establishing research-based computational curricula.


There is a gap between the (meager) research and classroom practice\cite{Yadav2015}. Computing education and research suffer from the lack of assessment instruments\cite{voogt2017effecten}. In their research \citeA{2010TewGuzdial}, argument the importance of validated assessment that could be widely applicable across tertiary curricular approaches. This need may be even more true for secondary education, where far less relevant research has been done in the first place. Also, research developments are not accessible to teachers (either they need a paid subscription, academic literature may be written incomprehensibly, or not practical as it can't be implemented in the classroom in a cost-effective manner). What is needed are validated assessments which are of high quality, sound, scientifically grounded, yet which can be implemented in a practical manner in a classroom setting.

With all these factors in place, what is needed for teachers and their PCK to adapt to, and successfully implement, the curriculum reform?
