\section{Significance}\label{sec:significance}


We hope this paper will provide input for future development of assessment instruments and strategies and contributes to a national dialogue about how to support teachers to effectively assess their students learning.

%\subsection{Study I. Relevant assessment design principles.}
\subsection{Study III. Bridging the gap of assessment knowledge.}
\subsubsection*{Practical significance}
Using the idea that assessments must be adapted and ‘recontextualized’ to fit classroom needs (Hermansen, 2014) as a starting point, the results of our research leads to a set of assessment templates, guidelines and examples which could help teachers understand the implications on the new curriculum, implement (aspects of) and adequately assess students’ knowledge in a cost-effective manner, yet fine-tuned to contexts relevant to their specific classroom setting. It enhances teachers’ PCK by supporting their ‘knowledge work’ (i.e. the activities which they carry out based on the knowledge from their professional practice).%, lowers workload and increases self-efficacy.

In addition, our research informs educational researchers and professional development programs about what types of assessments and instruments are suitable for testing algorithmic and computational thinking concepts and what knowledge and skills teachers need to be adept with in order to apply these assessments.

\subsubsection*{Scientific significance}

Assessment researchers need to better understand the design principles and psychometric properties of assessments that integrate computational thinking and problem solving with the fundamental concepts of computing science. CS education researchers wish to better understand the implications of widespread adoption of the CT learning perspective, including the implementation of reformed CS curricula and teacher professional development. Computer Science educators and policy makers want assessments that reliably measure the knowledge and abilities that are needed to engage in and support computer science learning in the classroom.




\subsection{Study IV. Variable features: untangling programming tasks.}
\subsubsection*{Practical significance}

The results of our research leads to a set of assessment tasks which could be used by teachers (of both secondary and tertiary CS education). The tasks help teachers identify which specific concept or strategy is problematic for a student. This basis can be used for determining variable features in programming tasks, and thus facilitating differentiation in the classroom. Tailored assessment could lead to shorter and faster tests (Zavala, 2018). Students not capable of completing a complex task could be given more fundamental tasks, while competent students could be given more challenging tasks. For example, a student not capable of understanding a selection in a loop could be then be given subsequent tasks on 1) selections, 2) loops, and 3) nesting to pinpoint what is specifically causing the struggle. The test can stop as soon as an adequate assessment of the student’s knowledge has been made.

As most research has been done in tertiary education (CS1 and CS2 courses) where assessments focus on a more advanced level of understanding and coincide with both higher developmental stage and cognitive load than is expected from students in secondary education, teachers could benefit from research focusing specifically on more fundamental and isolated assessment tasks.


\todo{This study is a part of a larger study in which a template for assessment tasks for secondary education are established to cater teachers on their mission to create qualitative assessment tasks adapted to their own specific classroom needs.}


\subsubsection*{Scientific significance}
The results of our research could lead to a set of assessment tasks which could be used by researchers to better investigate the understanding and alternative conceptions of students at both secondary and tertiary CS educational institutes. In particular, it is a novel step towards research specifically focused at secondary education and could highlight the specific differences between students at secondary and tertiary education.

The results of this study could be used as input for the conceptual assessment framework as a basis for determining variable features in programming tasks. This may contribute to the field of automated and/or adaptive testing (Zavala, 2018) and intelligent tutoring systems (Luxton-Reilly et al., 2018).


Future studies may include transforming the task into tasks easier to grade and analyze (such as multiple-choice assessment tasks, possibly combined with short-answer questions), tasks developed to specifically seek out common misconceptions, tasks for ‘creating’, and relating the assessment tasks to a CS-adequate taxonomy.

