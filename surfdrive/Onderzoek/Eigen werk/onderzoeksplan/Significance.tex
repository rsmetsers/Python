\section{Significance}\label{sec:significance}


The aim of work described in this thesis is to improve the assessment of Algorithms and Physical Computing in secondary education.
We investigate and establish student competency models for Algorithms and Physical Computing. We typify the PCK characteristics for these domains, and investigate and add assessment elements and guidelines designed to overcome difficulties faced by teachers. We hope this paper will provide input for future development of assessment instruments and strategies and contributes to an international dialogue about how to support teachers to effectively assess their students learning.

Generally teachers want to, and should, create assessments that are relevant for their students. Ag design pattern can aid teachers who have neither the time nor the expertise to indulge in the specific factors needed to measure the wide variety of aspects pertaining to both concepts and CT practices. As the fundamental concepts of both Algorithms and Physical Computing are new to the curriculum, a teacher's knowledge may still have to be further developed before a teacher can effectively create their own assessments which objectively measure the related concepts and skills. In addition to supporting assessment generation by means of a design pattern, templates, guidelines and con crete example assessments can help teachers determine if they are teaching the right concepts, at an adequate level and are endorsing the use of correct skills.

We review existing assessment strategies and investigate methods to efficiently and effectively bridge the assessment gap, as a part of PCK on Algorithms and Physical Computing and in particular the design of a digital artefact. We investigate and establish student competency models. We then use the design and implementation of assessment templates which support the design of families of assessment tasks to investigate how teachers can adapt these to create assessments specific to their classroom needs. This yields insight into teachers' PCK in these domains.






\subsection{Study I. Assessment of Algorithms}
\todo{see de raadt 2008 dissertation for his significance.. where he says teach, you say assess}
The first study focusses on assessing student's attainment level of computational practices with respect to Algorithms. %docenten worden indirect betrokken
%% pertains to investigating ways in which teachers can establish qualitative assessment tasks for the new Dutch curriculum.
%%\item The second is called "untangling programming tasks" and investigates a framework for decomposing and arranging programming (sub)tasks in a manner to specifically identify misconceptions in student learning, and variable features in programming tasks.
%


\subsubsection*{Practical significance}

The results of the first study leads to a progression pathway which integrates programming concepts with programming strategies (plans). The tasks created in the course of this research can help teachers (in secondary and tertiary CS education) identify which specific concept or strategy is problematic for a student. In addition, the established hierarchy of 'task variables', which are part of the design template (from the ECD Framework), can be used to vary the difficulty of tasks set.


The determination of the variable features of difficulty in programming tasks, facilitates differentiation in the classroom.  Traditionally, a teacher may attempt to establish a multiple-choice question with distracters to pinpoint the misconception, this has a few drawbacks. Firstly, it is very difficult to establish meaningful distracters, contrarily, students are sometimes even be provoked or put on the wrong track by distracters. Furthermore, if a student is struggling with two of these concepts, only one will be brought to light. In order to determine what the particular misconception is that the student is struggling with, requires confronting the student with different tasks. As assessments are created a priori, the tasks are not relevant to, or correspond with, the students level of understanding. Students not capable of completing a complex task could be given more fundamental tasks, while competent students could be given more challenging tasks.


For example, a student not capable of understanding a selection in a loop could be then be given subsequent tasks on 1) selections, 2) loops, and 3) nesting to pinpoint what is specifically causing the struggle. The test can stop as soon as an adequate assessment of the student's knowledge has been made. With the knowledge that the student could not answer the previous question, an extended question is not reasonable to ask. It does not yield any extra information about the student's level of attainment, and thus only bourdons the student's time and possibly lowering their self-efficacy. In turn, this last aspect is not constructive for the motivation (see section \ref{sec:qualityCriteria}). Contrarily, if a student fails to complete this task correctly, they should be given 3 subtasks, one for each of the concepts which it entails. Obviously, the student must complete 3 extra questions. But it does allow for a more precise assessment, pointing out exactly which concept the student is struggling with and what the student is proficient in. Such feedback could well be used for both formative (feedback on what to work on) and summative assessment. Tailored assessment could lead to shorter and faster tests (Zavala, 2018).



%Subsequently, a pilot was initiated to research the design template from the Assessment Implementation and Assessment Delivery layers. During the implementation stage, the model assessment tasks were translated into classroom-specific assessment tasks in the designated programming language. The task variables can be integrated to vary the difficulty of a task.


As most research has been done in tertiary education (CS1 and CS2 courses) where assessments focus on a more advanced level of understanding and coincide with both higher developmental stage and cognitive load than is expected from students in secondary education, teachers could benefit from research focusing specifically on more fundamental and isolated assessment tasks.


\todo{This study is a part of a larger study in which a template for assessment tasks for secondary education are established to cater teachers on their mission to create qualitative assessment tasks adapted to their own specific classroom needs.}


\subsubsection*{Scientific significance}
The results of the second study lead to a student competency model for computational practices related to Algorithms. The hierarchy of strategies (assessed through plans) are incorporated into a progression pathway.


This could be used by researchers to better investigate the understanding and alternative conceptions of students at both secondary and tertiary CS educational institutes. In particular, it is a novel step towards research specifically focused at secondary education and could highlight the specific differences between students at secondary and tertiary education.

The results of this study could be used as a basis for determining variable features in programming tasks. This may contribute to the field of automated and/or adaptive testing (Zavala, 2018) and intelligent tutoring systems \cite{LuxtonReilly2018}.


Future studies may include transforming the form of the assessment task into those easier to grade and analyze (such as multiple-choice assessment tasks, possibly combined with short-answer questions), tasks developed to specifically seek out common misconceptions.
%, and relating the assessment tasks to a CS-adequate taxonomy.


\subsection{Study II. Assessment of algorithmic concepts with recontextualization}
The second study is concerned with assessing recontextualization and transfer of algorithmic concepts, and typifying the PCK aspects that play a role in the assessment of this domain.

\subsubsection*{Practical significance}

The results of the second study leads to a student competency model about Algorithms. Following the ECD framework we establish variable contexts for assessing Algorithms. Selecting different contexts can help teachers to:
\begin{itemize}
\item choose contexts which are relevant and meaningful to their students,
\item vary tasks for the creation of new additional assessments, and
\item assess student's higher-order transfer capabilities for a new situation (or context).
\end{itemize}
We hope the results will provide insight on how to effectively fine-tune a design template to a teacher's own specific classroom needs, how to embed recontextualization into assessments, and as such informs effective classroom practice.


%Using the idea that assessments must be adapted and 'recontextualized' to fit classroom needs (Hermansen, 2014) as a starting point, the results of our research leads to a set of assessment templates, guidelines and examples which could help teachers understand the implications on the new curriculum, implement (aspects of) and adequately assess students' knowledge in a cost-effective manner, yet fine-tuned to contexts relevant to their specific classroom setting. It enhances teachers' PCK by supporting their 'knowledge work' (i.e. the activities which they carry out based on the knowledge from their professional practice).%, lowers workload and increases self-efficacy.
%
%In addition, our research informs educational researchers and professional development programs about what types of assessments and instruments are suitable for testing algorithmic and computational thinking concepts and what knowledge and skills teachers need to be adept with in order to apply these assessments.

\subsubsection*{Scientific significance}
In particular, this research involves and investigates teachers during recontextualizing of their assessments. This analysis yields insights on what teachers do, how they do it, what works and what they struggle with. This research aims to typify the characteristics of PCK for assessing Algorithms, and as such yield guidelines for teacher training to inform effective classroom practice.

%
%The aim of work described in this thesis is to improve the assessment of algorithms and programming in secondary education by:
%\begin{itemize}
%\item establishing a progression pathway wish integrates conceptual knowledge as well as computational practices,
%\item adding assessment elements to specifically pinpoint achievement levels of conceptual knowledge and computational practices independently, and
%\item analyzing and determining the teachers' PCK needed to deliver and implement such an assessment.
%\end{itemize}
%
%
%Assessment researchers need to better understand the design principles and psychometric properties of assessments that integrate computational thinking and problem solving with the fundamental concepts of computing science. CS education researchers wish to better understand the implications of widespread adoption of the CT learning perspective, including the implementation of reformed CS curricula and teacher professional development. Computer Science educators and policy makers want assessments that reliably measure the knowledge and abilities that are needed to engage in and support computer science learning in the classroom.
%
%


\subsection{Study III. Assessment of Physical Computing}

The third study is concerned with assessing Physical Computing concepts and design practices, and typifying the PCK aspects that play a role in the assessment of this domain.


\subsubsection*{Practical significance}
The results of the third study leads to a student competency model about Physical Computing and design of digital artifacts. We hope the results will help develop criteria for assessment and grading, and provide insight on how to effectively assess this domain.


\subsubsection*{Scientific significance}
The results of our research lead to a student competency model for both the computational concepts and the computational practices related to Physical Computing. The need for developing such a competency model has been emphasized in previous research \cite{mareen2018PhysComp}. This research involves and investigates teachers during assessment of Physical Computing. This analysis yields insights on what teachers do (and don't do), how they do it, what works, and what they struggle with. This research aims to typify the characteristics of PCK for assessing Physical Computing and design of digital artegacts, and as such yield guidelines for teacher training to inform effective classroom practice.


