\begin{abstract}

%Programming, where problem solving and coding come together, is cognitively
%demanding. Whereas traditional instructional strategies tend to focus on
%language constructs, the problem solving skills required for programming
%remain underexposed.
%
%
%In an explorative small-scale case study we explore a ''thinking-first''
%framework combined with stepwise heuristics, to provide students structure
%throughout the entire programming process.
%
%
%
%Using unplugged activities and high-level flowcharts, students are guided
% to brainstorm about possible solutions and plan their algorithms before
% diving into (and getting lost in) coding details. Thereafter, a stepwise
% approach is followed towards implementation. Flowcharts support novice
% programmers to keep track of where they are and give guidance to what they
%need to do next, similar to a road-map.
%
%
%
%High-level flowcharts play a key role in this approach to problem solving.
%They facilitate planning, understanding and decomposing the problem,
%communicating ideas in an early stage, step-wise implementation and
%evaluating and reflecting on the solution (and approach) as a whole.

\end{abstract}
