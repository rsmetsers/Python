

\section{Conclusion and future work}\label{sec:conclusion}

The concepts of problem solving, algorithmic thinking and development of
programming solutions are closely related. Algorithm
design and a structured manner for problem solving become indispensable when
dealing with complex problems and their solutions.

The results of our primary exploratory case study seem to support our hypothesis. 
Thinking-first and planning ahead are important first steps in problem
solving. It requires a particular focus, we believe best achieved away from
the (distracting) computer. Selection of the suitable plan seems to benefit
from explicit instruction and 'thinking' before 'acting'. Algorithmic
thinking skills and a bird's-eye view are needed to link plans together to
create an adequate solution.


An appropriate plan to solve a problem (despite of any construct-based errors
introduced during implementation) has far more potential of leading to a
correct solution, than an incorrect plan which is implemented bug-free. Most
construct-based errors are syntax errors which the compiler will complain
about. However, logical errors are far more difficult to detect and fix. Plan
implementation is a cognitively demanding process: design, implementation and
testing of subtasks takes place while simultaneously considering the
high-level goals.

Flowcharts, and a stepwise approach to problem solving, can aid novice
programmers to keep track of where they are and give guidance to what they
need to do next, similar to how a road-map helps navigate. High-level
flowcharts aid decomposition, subtask recognition and lightens the cognitive
load (minimizing the errors arising from cognitive overload) as each subtask
is dealt with independently.


The exploratory case study that we conducted was small-scale. The material has been improved according to the annotations in section
\ref{sec:improvements}. The Dutch version is being published as official secondary school material for the Dutch national curriculum, and includes formative and summative tests. The English version will be made available through the Greenfoot website. We plan to carry out a \emph{follow-up study} in the course of this year. Student and teacher feedback while using the improved version will be collected and analysed. Feedback and new suggestions for improvements are to be incorporated in the following iteration. In addition, the following \emph{extensions} will be made:

%We will
%continue to teach the course over the next few years, both in high-school and
%as a freshman course in university. In addition, several teachers both
%nationwide and internationally have indicated that they use the course
%material in their classes. It would be interesting to carry out a follow-up
%study in which more data can be collected and analysed, yielding new
%suggestions for ongoing improvement of CS education.
