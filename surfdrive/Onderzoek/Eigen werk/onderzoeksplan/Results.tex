\section{Results}\label{sec:results}



%For this research we focus specifically on the level of understanding (the reading and tracing), and leave creating for further research. This set of assessment tasks could be used to adequately determine the mastery of a student for a specific aspect (for either a fundamental concept or a plan composition problem) and discover any alternative conceptions. In addition, it facilitates the identification of variable features.


%The assessments created by the teachers along with samples of their graded work will be analyzed in Atlas. In addition to semi-structured interviews, and teachers' notes on their thoughts and experiences during implementation will be used to investigate characteristics of PCK:
%\begin{itemize}
%\item experience using the design template (i.e., how much time they spent, if they encountered any problems, if they felt it was useful, if they felt they had sufficient knowledge and skills, what they used, what they didn't use)
%\item how they approached the task (i.e., how they started, what they did to overcome any hurdles),
%\item the quality criteria aspects discussed in section \ref{sec:qualityCriteria} (i.e. how satisfied are they about their own assessment),
%\item any suggestions for improvements.
%\end{itemize}



%This study starts with a literary review to inventory what is known about challenges that teachers face regarding assessment (as a part of PCK) in general, and pertaining to algorithms and algorithmic thinking in particular.
%
%
%The study continues with a document analysis of related work pertaining to assessment of CS concepts and skills. Research on existing assessment instruments and relevant international curricula are reviewed. To select the curricula, we first listed those curricula which have been referred to in recent publications. The next selection criteria pertained to the relevancy of the learning objectives. Those must coincide with the reformed Dutch curriculum, with a particular focus on algorithms or problem solving, as well as explicitly embedding Computational Thinking as a core competency. In the third selection, relevant supportive documentation was considered, such as progression pathways (CAS, corresponding to the national curriculum in the U.K.) or specific description of the KSA outcomes and mappings to their corresponding curricula (ECS and AP CS Principles as specifications of the U.S. CSTA, and AQA as specifications of the England national curriculum). This narrowed down to the following short-list of curricula: U.S. CSTA, the Israel nationwide CS curriculum, National curriculum for computing England, and the NCEA Digital Technologies curriculum in New Zealand. From these curricula, sample assessment tasks and the way in which they elicited evidence in relation to their KSAs was analyzed together with any available research generally describing the curriculum or specifically describing the development of assessments. In addition, relevant (inquiry based) tasks from Process Oriented Guided Inquiry Learning in Computer Science (CS-POGIL), the International Baccalaureate (IB) program, and the Computer Science Field Guide \cite{CSFieldGuide} were reviewed.
%
%
%Similar to the ECS, we use the structured ECD approach to establish a design template for assessments which can be fine-tuned by teachers to adhere to their own specific classroom needs. This structured approach results in a clear link between potential observations (evidence of students' knowledge or skills), evaluation procedures, and measurement models (see section \ref{sec:ECD}).\todo{evt rubrics etc}
%
%
%The design pattern is piloted in two stages. The results from literary review will be triangulated with teacher interviews and observations during the implementation and delivery phases in which teachers use templates to implement and deliver their customized assessments. Via qualitative analysis (teacher interviews, student think-aloud sessions, student self-evaluation, assessment task analysis, scoring process analysis) we seek to determine which characteristics in the design-patterns are feasible as PCK-enablers, and which aspects require refinement or further investigation (see also table \ref{table:MappingCriteriaMethod}). In an iterative process, the results are analyzed and evaluated, and the assessment instrument adjusted and refined. This process is repeated twice, first in a small-scale pilot, and then the candidate template assessment tasks will be deployed at multiple institutions for testing on a larger scale. We subsequently report our findings.


\subsection{Possible Threats to Validity}
The teachers who participated did so voluntarily. In addition, they were part of a group of teachers involved in creating lesson materials for the new curriculum. These teachers have been selected carefully through a committee. It does not reflect a fair sample of the average computer science teacher.


\subsection{Future Work}

\begin{itemize}
\item Recontextualization:
Eliciting evidence for  \citeA{oers2004recontextualization} first step (Intended focus: Viewing the situation or problem from a perspective in which the inscription makes sense.) explicitly and separately from the model the students create.

\end{itemize}

