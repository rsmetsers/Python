\section{Aim of the thesis}\label{sec:aim}
All-in-all, little research has been done on the effectiveness of the assessment materials available \cite{Yadav2016}. Recent published work based on CT focusses on primary education or middle school \cite{LyeKoh2014}. Research based on (the assessment of) CS concepts has a focus on tertiary education (such as \cite{McCracken2001},\cite{2010TewGuzdial}). More specifically, we know little about how teachers are to assess specific CS concepts or computational practices pertaining to Algorithms and Physical Computing in secondary education.

The aim of this thesis is to contribute to the knowledge about assessment pertaining to the domains of algorithms, physical computing and their related computational practices and typifying which PCK it demands from teachers to assess these domains.
%and what PCK is requires/demands/involves from teachers
%knowledge domains of algorithms, physical computing and the related computational practices.
%en vertalen we naar...

\subsection*{Algorithmic task difficulty}
Section \ref{sec:researchAssProgramming} describes that despite efforts from the research community, creating programming assessment tasks that measure what we wish to measure, especially with respect to difficulty, remains a challenge. It was outlined that several researches attempt to classify tasks according to task-oriented features, however do so without isolating the elementary concepts from the strategies, and as a result try to assess (nested) concepts and (problem-solving) strategies simultaneously. Determining task difficulty could be facilitated by clearly separating conceptual knowledge from computational practices (as proposed by \citeauthor{BrennanResnick2012}). As proposed by \citeauthor{deRaadt2009teachingPlans}, plans could be integrated into tasks order to assess strategies. To our knowledge, there has not yet been an attempt to systematically untangle assessment tasks by focussing on computational practices and in doing so, isolating and identifying not only elementary programming concepts but also algorithmic components \cite{deRaadt2008}. The focus lies specifically on the fundamental CS concept of algorithms and the closely coupled algorithmic thinking (as a part of CT) and problem solving skills. The choice for a focus on algorithms is not an arbitrary one. With the curriculum reform, this fundamental concept has become much more theoretical. In addition, algorithmic development and Computational Thinking skills go hand-in-hand and few teachers of aware of the recent research in this area, let alone have any idea on how to assess this.

This leads us to our first question about how variable features of \emph{task difficulty}, specifically related to computational practices, can be identified, accounted for and integrated into the assessment of \emph{Algorithms}.



\subsection*{Task contexts}
%    \item 1b. Algorithms: How can the variable features of task contexts be identified, accounted for and integrated into an assessment?
Problem solving involves a continuous process of abstracting and choosing new contextualizations within new situations \cite{oers2004recontextualization}. That transfer and the process of abstraction (as an aspect of CT) is perceived as an important higher-order skill has been discussed in section \ref{sec:AlgProblemSolving}. In addition to dealing with diversity and enhancing inclusion, providing students with meaningful tasks tailored to their specific needs constitutes one of the quality criteria for assessments \ref{fig:AssQualityCriteria}. Despite attempts by researchers in different domains, they have not yet been able to pinpoint the mechanisms behind transfer \cite{oers2004recontextualization}. Though the mechanisms may not yet be crystal clear, understanding which types of contexts can be related and how abstraction can be used to enhance transfer could lead to more attractive and relevant tasks for the students, in addition to a larger and more flexible pool of assessment questions for the teachers to draw upon.

This leads us to the second question about how variable features of \emph{task contexts} be identified, accounted for and integrated into the assessment of \emph{Algorithms}.



\subsection*{Digital artefact design skills}
%    \item 1c. Algorithms: How can attainment levels of design skills be identified, elicited, and measured?
%    \item 2a. Physical Computing: How can attainment levels of design skills be identified and measured?
Globally, reformed computer science curricula have made an educational shift in focus from 'user' to 'creator'. With it, this movement has placed emphasis on the higher-order Computational Thinking skills involved with solving complex problems \cite{Wing2006}. Thus, skills used during the design and development of a digital artifacts have become predominant. These skills include evaluation of quality, such as correctness and efficiency, and weighing options by means of research and experimentation. Current research is focussed merely on concepts. \citeA{LyeKoh2014} express that only assessing concepts is not enough and additionally embrace computational practices as distinguished by \citeauthor{BrennanResnick2012}. Understanding how to deliberately put students into situations (with challenges or tasks) which elicit these skills is imperative to adequately measure attainment levels. In their review study \citeA{voogt2017effecten} discuss how computing education research in primary and secondary education struggles with a lack of reliable, validated, and theoretically grounded CT assessment instruments. We focus specifically on the design of software and Physical Computing.

This leads us to the third question about how the attainment levels of \emph{design skills} for digital artifacts (specifically physical computing) can adequately be identified, elicited, and measured.



\subsection*{Attainment levels in Physical Computing}
%    \item 2b. Physical Computing: How can knowledge of concepts of Physical Computing (embedded systems) be identified and measured?
Physical Computing is a newly emerging field in computer science education.

This leads us to the fourth question, about how the knowledge of \emph{concepts} of \emph{Physical Computing} (embedded systems) can be identified and measured.



\subsection*{Assessment PCK of Algorithms and Physical Computing}
%    \item 3. PCK:  What characterizes the PCK of Physical Computing, Algorithms and computational practices, specifically the PCK component pertaining to effective assessment?
As described in the background section, teachers find it difficult to assess student learning in computer science \cite{yadav2016pck}. There is a gap between the (lean) research and classroom practice \cite{Yadav2015}. In their research \citeA{2010TewGuzdial}, argument the importance of validated assessment that could be widely applicable across tertiary curricular approaches. This is even more so for secondary education, where far less relevant research has been done in the first place.



  %In addition, research developments are not accessible to teachers. They need a paid subscription, and academic literature may be written in an incomprehensible manner, or not practical as it can't be implemented in the classroom in a cost-effective manner (for example think-aloud sessions).



To ensure creating tasks meaningful to students (in addition to dealing with diversity and promoting inclusion of all types of students), teachers will and should create assessments tailored specifically to their classroom needs. However, they lack specific PCK to do so for Algorithms and Physical Computing. As the more conceptual topics and their corresponding student skills are new, teachers may be concerned about not matching the expectations of the curriculum's standards. The introduction of the new computing curriculum could benefit from understanding how to support teachers whilst creating qualitative knowledge and competency based assessments. Not only is creating a qualitative assessment acknowledged as a difficult task, in the current situation, teachers must first acquire and internalize knowledge about the new concepts. Teachers could benefit from assessment templates which they can fine-tune according to their own specific classroom needs. This could help establish qualitative assessments which adhere to the curricular objectives as well as being aligned to their teaching and learning activities.  It would be interesting to understand what characterizes and typifies PCK for effective assessment. This could provide insight on knowledge patterns and (evaluation) strategies with regards to assessment. This insight is three-fold:
\begin{itemize}
\item typify PCK of Algorithms and Physical Computing with respect to assessment,
\item establish good practices of assessment principles (such as eliciting evidence on proficiency from students) and fine-tuning assessment design templates, identifying its significance and implications, and
\item inform required professional development.
\end{itemize}

This leads us to the fifth question, namely what characterizes the \emph{PCK of Physical Computing and Algorithms}, specifically the PCK component pertaining to effective \emph{assessment}.


%We report on our finding and propose a set of  lessons learned for similar %future initiatives. and PD initiatives


%We propose, pilot and review an assessment design pattern established via Evidence-Centered Design. %We report on the educational implications of our findings with respect to the design pattern. In %addition, we investigate the methodology used and propose a set of lessons learned for similar %future initiatives.


%We explore the use of flowcharts in combination with unplugged
%activities, instructing the use of `plans' and Greenfoot's visualisation techniques as an
%approach to alleviating some of the programming difficulties.


\subsection{Research Questions}


%The main research question is: \emph{How can teachers be empowered to adequately assess the reformed CS curriculum topics of algorithms and physical computing and the related computational practices?}
To summarize, the following research questions will be addressed:


\begin{enumerate}
\item \textbf{(Algorithms)}: How can the variable features of task difficulty related to computational practices be identified, accounted for and integrated into the assessment of Algorithms?
\item \textbf{(Algorithms)}: How can the variable features of task contexts be identified, accounted for and integrated into the assessment of algorithms?

\item \textbf{(Physical Computing)}: How can attainment levels of design skills for digital artifacts (specifically physical computing) be identified, elicited, and measured?
\item \textbf{(Physical Computing)}: How can knowledge of concepts of Physical Computing (embedded systems) be identified and measured?


\item \textbf{(PCK)}: What characterizes the PCK of Physical Computing and Algorithms, specifically the PCK component pertaining to effective assessment?

\end{enumerate}


%Our approach with a strong focus on algorithmic design and thinking-first supports the problem % solving process during programming.
%
% How does an approach in which
%students are encouraged to iteratively think-and-design (using flowcharts and unplugged)
%prior to implementation (in code), guide and support students in the problem solving
%process of programming?

%The main research question is: \emph{How can teachers be empowered to adequately assess the reformed CS curriculum topic of algorithms?}
%
%
%In order to answer this question, the following subquestions are addressed:
%\begin{enumerate}
%\item What are the relevant assessment design principles?
%\item How to elicit and judge evidence of relevant knowledge and skills?
%\item How to effectively bridge teachers' assessment knowledge (as part of PCK) gap?
%\item How can the variable features of task difficulty be identified and accounted for and integrated in an assessment?
%
%
%\end{enumerate}

