\section{Aim of the Study}\label{sec:aim}

\todo{Erik: Tekortkomingen omzetten in een wetenschappelijk vraag: op welke manier kan hierarchie…}

We review existing assessment strategies and investigate methods to efficiently and effectively bridge the assessment gap (as a part of PCK). The goal is to investigate the design and implementation of assessment templates which support the design of families of assessment tasks. Particularly, how teachers can adapt these to create assessments specific to their classroom needs. The focus lies specifically on the fundamental CS concept of algorithms and the closely coupled algorithmic thinking (as a part of CT) and problem solving skills. The choice for a focus on algorithms is not an arbitrary one. With the curriculum reform, this fundamental concept has become much more theoretical. In addition, algorithmic development and Computational Thinking skills go hand-in-hand and few teachers of aware of the recent research in this area, let alone have any idea on how to assess this.


While deploying a design pattern, we were interested in researching whether, and to which extent, such a tool supports a teacher in creating an assessment tasks that is tailored to individual classroom needs. Particularly the teacher’s perception of the tool was of interest, such as its ease-of-use, use-for-purpose. In addition, we were interested in resulting assessment itself. Aspects of interest are quality (such as validity, objectiveness, fairness)


\todo{<SEE ALSO 2016 constructing_assessment_tasks_Science SRI>}


The goal is to support teachers in the development and implementation of their own classroom specific assessments. The resulting design pattern can aid teachers who have neither the time nor the expertise to indulge in the specific factors needed to measure the wide variety of aspects pertaining to both algorithms and CT. As the more conceptual topics and the required student skills are new, teachers may be concerned about not matching the expectations of the curriculum's standards. In addition to supporting assessment generation by means of a design pattern, example assessments can help teachers determine if they are teaching the right concepts, at an adequate level and are endorsing the use of correct skills.

The focus lies on the summative school exam, as prescribed by the Dutch national curriculum. It includes both cognitive factors (algorithmic concepts) and noncognitive factors (CT).  Particularly, the non-cognitive skills are often not easily measurable, and as a result, educators can have a difficult time assessing these. In addition, as the fundamental concepts of algorithms are new to the curriculum, a teacher’s knowledge may still have to be further developed before a teacher can create their own assessments to objectively measure the related cognitive skills. Although generally teachers want to, and should, create assessments that are relevant for their students themselves, at least initially, it would be helpful to be given a template, guidelines and concrete examples on how to do that.


The domain of interest is programming (domain D), relevant algorithms (domain B) and related skills (domain A), as specified in the Dutch curriculum specification. Together these entail program-relevant concepts and (Computational Thinking) skills.


As outlined in the background section, many researches show an attempt to classify tasks according to task-oriented features, however do so without isolating the elementary concepts from the strategies, and as a result try to assess (nested) concepts and (problem-solving) strategies simultaneously. It is our opinion this type of research could be mediated by using tasks which clearly separate concepts from plans. The starting point is to use a bottom-up approach to identify tasks that assess concepts individually. Then, systematically apply merging, abutment and nesting to individual concepts in combination with a top-down plan decomposition approach (similar to the work of Luxton-Reilly et al., 2018 and de Raadt, 2009) to establish a (hierarchical) set of assessment tasks for more complex plans and programming strategies. For this research we focus specifically on the level of understanding (the reading and tracing), and leave creating for further research. This set of assessment tasks could be used to adequately determine the mastery of a student for a specific aspect (for either a fundamental concept or a plan composition problem) and discover any alternative conceptions.

Our approach is novel because, to our knowledge, there has not yet been an attempt to systematically untangle assessment tasks and in doing so, isolating and identifying elementary concepts and plans.



The aim of the study is two-fold. We investigate methods for teachers to develop qualitative summative assessment instruments for computational thinking and algorithms for CS secondary education. We also explore, pilot, and report on a method for effectively determining students' programming levels of mastery and identifying specific misconceptions.

%We report on our finding and propose a set of  lessons learned for similar %future initiatives. and PD initiatives


%We propose, pilot and review an assessment design pattern established via Evidence-Centered Design. %We report on the educational implications of our findings with respect to the design pattern. In %addition, we investigate the methodology used and propose a set of lessons learned for similar %future initiatives.


%We explore the use of flowcharts in combination with unplugged
%activities, instructing the use of `plans' and Greenfoot's visualisation techniques as an
%approach to alleviating some of the programming difficulties.


\subsection*{Research Questions}

%Our approach with a strong focus on algorithmic design and thinking-first supports the problem % solving process during programming.
%
% How does an approach in which
%students are encouraged to iteratively think-and-design (using flowcharts and unplugged)
%prior to implementation (in code), guide and support students in the problem solving
%process of programming?

The main research question is: \emph{How can teachers be empowered to adequately assess the reformed CS curriculum topic of algorithms?}


In order to answer this question, the following subquestions are addressed:
\begin{enumerate}
\item What are the relevant assessment design principles?
\item How to elicit and judge evidence of relevant knowledge and skills?
\item How to effectively bridge teachers' assessment knowledge (as part of PCK) gap?
\item How can the variable features of task difficulty be identified and accounted for and integrated in an assessment?


\end{enumerate} 

